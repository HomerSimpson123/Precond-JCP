% Elsevier article
%\documentclass[authoryear,preprint,review,12pt]{elsarticle}
\documentclass[final,3p,times,twocolumn]{elsarticle}

% packages
\usepackage{amssymb}

% TODO: remove me later
\usepackage{blindtext}

% journal
\journal{Journal of Computational Physics}

\begin{document}

\begin{frontmatter}
%% Title
\title{Preconditioning for Combustion Problems with Detailed Chemical Mechanisms}

%% MAH
\author[uofu]{M.A.~Hansen\corref{cor}\fnref{mah}}
\ead{mike.hansen@chemeng.utah.edu}
\cortext[cor]{Corresponding author}

%% JCS
\author[uofu]{J.C.~Sutherland\fnref{jcs}}
\ead{james.sutherland@chemeng.utah.edu}

%% UofU
\address[uofu]{University of Utah, Department of Chemical Engineering, 201
President's Circle, Salt Lake City, UT 84112}

%% Abstract
\begin{abstract}
\blindtext
\end{abstract}

%% Keywords
\begin{keyword}
Preconditioning \sep Dual timestepping \sep Chemically reactive flow
\end{keyword}
\end{frontmatter}

\section{Introduction}
% Introduction
% --------------
% + Introduce the simulation problem = stiffness
%	- Low-Mach
%	- Combustion
%	- kinetic vs acoustic vs advective vs diffusive time scales
% 
% + Introduce the generic solution methodology = prec. dual timestepping
%	- Started with the use of timestepping methods for the incompressible and low-Mach number flows 
%	- Preconditioning introduced or modified acoustic waves to accelerate convergence
%	- Dual timestepping used to obtian time-accuracy
%	- Compressible equation set used for all flows
%	- Chemistry preconditioning has seen very limited effort
%	- Venkateswaran, 1995 and Sankaran, 2007 (Joe's paper)
%	- Elmahi, 2008 develops combustion sim. but does not precondition chemistry
% 
% + Introduce the specific contribution = our new preconditioners
%	- We develop new strategies, not just new parameters
%	- Combustion of hydrogen in a 0-D reactor, full mechanism
%	- Compare with our implementations of prior preconditioners
%
Time-accurate integration of low-Mach combustion is difficult because of
the inherent stiffness between kinetic, acoustic, advective, and diffusive time
scales. Tight coupling between high- and low-frequency modes places severe
constraints on traditional solution techniques, particularly when employing
highly-detailed chemical reaction mechanisms. 

Preconditioned dual timestepping has enjoyed success as a tool for
simulation of all-Mach number flows(citations here).

Extension of this methodology to chemically reactive flows has seen little
development. \citet{Venkateswaran1995} unsuccessfully attempted to apply
time-derivative scaling of the species equations. \citet{Sankaran2007} consider
the eigenvalues of a one-step combustion mechanism in developing their
preconditioning strategy. Their results show success in accelerating
convergence of a 16-species, 12-reaction mechanism for methane-air combustion.
In this paper we develop and test new preconditioning heuristics for the
chemical source terms. We compare results against the preconditioners of
\citet{Venkateswaran1995, Sankaran2007}.

\section{Conclusions}
\blindtext
\blindtext

\section{Acknowledgement}
\blindtext

\section{References}
\bibliography{ref.bib}
\bibliographystyle{elsarticle-num-names}



\end{document}
